\documentclass{mall}

\newcommand{\version}{Version 1.0.1}
\author{Love Arreborn, \url{lovar063@student.liu.se}\\
  Tove Winddotter, \url{tovwi303@student.liu.se}}
\title{Gruppkontrakt}
\date{2022-09-13}
\rhead{}

\begin{document}
\projectpage

\section{Hur man samarbetar}

\subsection{Arbetstider}

- Om inget annat nämns så gäller arbetstiderna vardagar mellan 08-17.

- Möten kommer ta form som ett fysiskt veckomöte, samt dagliga uppdateringsmöte.

\subsubsection{Interna deadlines}
- Arbetet skall utföras i god tid innan deadline, med god marginal för att hinna genomföra korrigeringar om behov uppstår. Arbetsgruppen vill kunna hålla en marginal av 2-4 arbetsdagar i förväg till projektets deadlines - dels för att säkerställa att vi håller en god takt under hela projektet, men även för att ta höjd för eventuella problem eller buggar som kan uppstå under arbetets gång.

\subsection{Arbetsfördelning}
Projektets arbete kommer utföras tillsammans och med individuella ansvar för vissa områden. Medan bägge deltagare kommer att vara delaktiga i samtliga moment så kommer visst huvudansvar att delas upp på individnivå.

- Love kommer ta huvudansvar för den estetiska designen av webbsidan, det vill säga webbplatsens HTML och CSS. Den bakomliggande strukturen av webbplatsen kommer att redovisas för att säkerställa att även Tove kan genomföra justeringar eller korrigeringar vid behov.

- Tove kommer ta huvudansvar för att skriva ned vår dokumentation i LaTeX. Innehållet i dokumentationen tar gruppen gemensamt ansvar för under möten.

- Till en början av programmeringsfasen skall samarbete vara en prioritet, för att ge mer programeringskundskap till bägge deltagare.

- Under hela projektets gång kommer gemensamt ansvar att tas för programmeringen i Python.

- Vid mindre arbete med specifika delar av projektet kommer man att fylla i Jira innan, under och efter delen är klar för att tydliggöra för partnern vad som gjorts, och vad som är kvar att göra.

\subsection{Kommunikation}
Kommunikation kommer främst att ske i person eller Discord.

- Vad varje tillfälle som projektet arbetas på ska en kort summering finnas tillgänglig för samtliga i gruppen.

- Om inget annat nämns så gäller arbetstiderna vardagar mellan 08-17 men frågor och kommentarer får skickas via direktmeddelanden på Discord.

- Minst ett fysiskt möte i veckan, om inget annat nämns.
- En daglig uppdatering på ca 15 min med valfri kommunkationskälla. Där man täcker:
      - Vad som skall göras under dagen
      - Vad som har gjorts sedan sista
      - Eventuella brister

\subsection{Framförande av åsikter}
- Om en gruppmedlem känner sig inte hörd så har denne en plikt att ta upp det, och vi ska båda ha förståelse för detta om detta väl skulle uppstå.

- Vid andra obligationer under vardagar 8-17 skall gruppen meddelas i tid, och den frånvarande individen ansvarar själv för att arbeta ikapp den tid som har missats.

\subsection{Hänsyn och förståelse}

- Vid planering definieras 17 som 17:15 (som akademisk kvart), och 17:00 menar prick denna tid.

\subsubsection{Stöd}

- Alla frågor förtjänar svar. Även om svaret inte kan ge en lösning på frågan.

- Vid grov irritation på projekt eller projektparter är det lämpligt att ta en paus och sedan återsluta efter 30 minuter.

- Gruppen godkänner arbetstiderna, men tillåter även att medelanden genom Discord får skickas när som helst. Bägge deltagare är medvetna och accepterar att svaret inte måste komma först nästkommande arbetsdag.

\end{document}
