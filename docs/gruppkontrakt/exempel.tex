\documentclass{mall}

\usepackage{xcolor}
\usepackage{blindtext}
\usepackage{geometry}
 \geometry{a4paper, total={170mm,257mm},left=20mm,top=20mm,}
\usepackage{tabularx}
\usepackage[hmargin=1cm]{geometry}

\newcommand{\version}{Version 1.0}
\author{Love Arreborn, \url{lovar063@student.liu.se}\\y
  Tove Winddotter, \url{tovwi303@student.liu.se}}
\title{TDP003 -- Projektplan}
\date{2022-09-22}
\rhead{}

\begin{document}
\projectpage

\title{}{Projektplan}
\pagebreak
\section{Revisionshistorik}
\begin{table}[ht]
\begin{tabularx}{\textwidth}{|l|X|l|}
\hline
Version & Revisionsbeskrivning                      & Datum \\ \hline
0.1     & Skapat första utskick av dokument        & 14/9       \\ \hline
1.0     & Utkast av första version av projektplanen & 22/9       \\ \hline
1.1     &                                           &       \\ \hline
\end{tabularx}
\end{table}

\section{Introduktion till Projekt: Egna datamiljöer}

Projektets mål är att skapa en portfolio-programvara i Python, som sedan ska kunna vara till hjälp med dokumentation av kommande projekt. Portfolion ska skapas som en webbplats. Webbplatsen kommer att bestå av två huvuddelar: ett datalager och presentationslager. Datalagret kommer sedan att specificeras med ett API (Application Programming Interface) som kommer att användas som en modul med specifikationer. Detta kommer även användas av presentationslagret för att skicka information till datafilen. Presentationslagrets uppgift är att avbilda den informationen som användaren har begärt på ett fint och tydligt sätt, med specifikationer från datalagret. Detta görs genom att presentationslagret genererar en HTML-kod som ska sänds till användaren.

Projektet kommer även inomfatta ett delmoment där en installationsmanual skapas. Denna manual är skapad så att andra ska kunna installera och använda portfolion.


\subsection*{Datalagret}
Datalagrets arbetsuppgift är att hantera datan i systemet, och specificeras med ett API. Datalagret kommer att lagras i en JSON-fil (JavaScript Object Notation).
Datalagret har fler krav utöver API som följer:

- Systemet ska kunna hantera följande information om ett projekt: projektnamn, projekt-ID-nummer, startdatum, slutdatum, kurskod, kursnamn, kurspoäng, använda tekniker, kort beskrivning, lång beskrivning, liten och stor bild, gruppstorlek och en länk projektsida. Projektnamn och projekt-id är obligatoriska, övriga fält kan lämnas tomma.

- Projekt-ID ska vara ett unikt heltal för varje projekt.

- Varje projekt kan ha en sekvens av tekniker angivna.

- Sökning ska kunna göras på godtycklig projektinformation, samt ska kunna utföras med flera valda tekniker.

- Sortering ska kunna göras på ett fält, i stigande och fallande träffordning.

- Man ska kunna filtrera utifrån använda tekniker i sökningen.

{Allt ovanstående ska fungera tillsammans, så att man kan söka på ett sökord, filtrera till vissa tekniker och sortera söklistan i en viss ordning samtidigt.}
\\

- Data lagras i en JSON-fil (JavaScript Object Notation) vid namn data.json.

- Filen ska lagras med UTF-8 teckenkodning.

- Data läggs till i JSON-filer manuellt (eller av andra verktyg) i systemet.

- Förändring av data.json ska slå igenom direkt i systemet utan omstart av webbserver.
\\

Tekniker som behövs till datalagret är Python3 och JSON.

\subsection*{Presentationslagret}

 Presentationslagret kommer med hjälp av information från en JSON-databas av datalagret, presentera projekt.
Presentationslagret skall bestå av fyra sidor:

- Index är förstasidan innehåller en kort personlig presentation.

- List kommer att visualisera projekt, sorterade efter använda programmeringstekniker.

- Projekt är en dynamisk sida som informerar mer detaljrik information om ett valt projekt. Med alla projekt tillgängliga för detaljvisning.

- Söksida är sidan som låter användaren se en lista av alla projekt samt sökfunktionalitet för att filtrera bland befintliga sökfilter.

Tekniker som behövs till presentationslagret är HTML/CSS, Flask, JavaScript och Jinja.

\pagebreak

\subsection*{Metodik}

 Schema för arbetet kommer att läggas upp från vecka till vecka, och arbetet kommer ske både tillsammans till stor del, men även parallellt för de mindre arbetsuppgifterna samt korrekturläsningar. Ett fysiskt veckomöte hålls 08:15 varje torsdag, samt att daglig uppdatering sker antingen digitalt eller fysiskt.

 Tidsplanen av veckorna under baseras på de redan schemalagda arbetstiderna på campus.

Alla versioner av projektets kod kommer att finnas tillgängligt via Git. Vid mindre arbete med specifika delar av projektet kommer man att fylla i Jira innan, under och efter delen är klar för att tydliggöra för partnern vad som gjorts, och vad som är kvar att göra.

 Arbetet skall utföras i god tid innan den kursens schemalagda hårda deadlines, med god marginal för att hinna genomföra korrigeringar om behov uppstår. Detta ger interna deadlines som projektgruppen har angivit på egen hand, och dessa kommer vara milstolpar att jobba utefter. Interna deadlines kommer att minska risker, såsom eventuella problem eller buggar som kan uppstå under arbetets gång som behöver åtgärdas. Hårda deadlines kommer framöver att markeras som rött, och interna deadlines markeras som blått i tidsplaneringen.

\subsection*{Projektöversikt}
 Samtliga datum i tabellen nedan märker ut hårda deadlines för de olika delmonenten i projektet.

\begin{table}[h]
\begin{tabularx}{\textwidth}{|l|X|l|l|}
\hline
Deadline & Delmoment av projekt                                                                            & Avsatt tid & Faktisk tid \\ \hline
8/9      & Gruppkontrakt inlämnad                                                               & 4h            & 4h              \\ \hline
13/9     & Tidsplanering inlämnad                                                               & 3h            & 3h            \\ \hline
16/9     & LoFi-prototyp skapad och inskickad                                                   & 5h          & 6h             \\ \hline
22/9     & Första utkastet av projektplanen klar och inlämnad                                   & 7h           & 6h             \\ \hline
22/9     & Första versionen av den gemensamma installationsmanualen                             & 5h           & 2h            \\ \hline
29/9     & Sammanställning av den gemensamma installationsmanualen                             &            &              \\ \hline
30/9     & API/Datalager färdigställt och godkänt av assistent                                  &            &              \\ \hline
13/10    & Fungerande prototyp av hemsidan som kan samspela med datalagret &            &              \\ \hline
13/10    & Portfolio tillgänglig via openshift                                                  &            &              \\ \hline
13/10    & Första versionen av systemdokumentationen inlämnad                                   &            &              \\ \hline
20/10    & Testdokumentationen inlämnad                                                         &            &              \\ \hline
20/10    & Individuellt reflektionsdokument inlämnat                                            &            &              \\ \hline
20/10    & Eventuella brister i systemdokumentationen korrigerade                               &            &              \\ \hline
\end{tabularx}
\end{table}

\pagebreak

\section{Tidsplanering} \\

\hphantom{$\bullet$} \textbf{Vecka 36}\\

Hård deadline
\begin{itemize}
\color{red}
    \item 8/9 Gruppkontrakt inlämnad
\end{itemize}

\begin{table}[h]
\begin{tabularx}{\textwidth}{|l|X|l|l|l|}
\hline
Aktivitet     & Beskrivning                            & Prioritet  & Avsatt tid & Tid \\ \hline
Gruppkontrakt & Skriva och lämna in ett gruppkontrakt  & Hög   & 4h          & 4h   \\ \hline
Tidsplan      & Skapa och påbörja tidsplanering        & Medel & 3h          & 3h  \\ \hline
\end{tabularx}
\end{table}

Kommentar: Komplettering på Gruppkontraktet inkom den 13/9, och korrigeras även detta datum. Ny version skickade och blev godkänd den 16/9.
\pagebreak

\hphantom{$\bullet$} \textbf{Vecka 37}\\

Intern deadline
\begin{itemize}
\color{blue}
    \item 16/9 Projektplan första version klar
\end{itemize}

Hård deadline
\begin{itemize}
\color{red}
    \item 13/9 Lämna in tidsplan

    \item 16/9 LoFi-prototyp inlämnad
\end{itemize}

\begin{table}[h]
\begin{tabularx}{\textwidth}{|l|X|l|l|l|}
\hline

Aktivitet    & Beskrivning                              & Prioritet  & Avsatt tid & Tid \\ \hline
Tidsplan     & Avsluta och lämna in tidsplan            & Hög   & 2h         & 2h     \\ \hline
LoFi         & Skapa och lämna in LoFi-prototyp         & Hög   & 3h         & 5h     \\ \hline
Projektplan  & Skapa ett första utkast av projektplanen & Hög   & 4h         & 3h    \\ \hline
Egen studier & Läs om projektet och kurslitteratur      & Hög   & 5h         &     \\ \hline
Manual       & Börja skriva på en installationsmanual   & Medel & 2h         & 2h    \\ \hline

\end{tabularx}
\end{table}

Kommentar: Komplettering på LoFi-prototypen inkom den 20/9, och korrigerades dagen efter. Ny version skickas in till assistenten för rättning.
Tidsplanen skickades in den 16/9 och behövde inga kompletteringar.

\pagebreak

\hphantom{$\bullet$} \textbf{Vecka 38}\\

Intern deadline
\begin{itemize}
\color{blue}
    \item Installationsmanual ska vara färdig
\end{itemize}

Hård deadline

\begin{itemize}
\color{red}
    \item 22/9 Första utkast av projektplan inlämnad

    \item 22/9 Skicka in
installationsmanual
\end{itemize}

\begin{table}[h]
\begin{tabularx}{\textwidth}{|l|X|l|l|l|}
 \hline
Aktivitet    & Beskrivning                                                                                              & Prioritet  & Avsatt tid & Tid \\ \hline
Projektplan  & Slutföra och skicka in projektplan                                                                       & Hög   & 3h         & 2h    \\ \hline
Lofi         & Slutföra och skicka in LoFi-prototypen                                                                   & Hög   & 2h         & 1h     \\ \hline
Manual       & Slutföra installationsmanualen & Hög   & 3h         & 1h   \\ \hline
Datalagret   & Påbörja datalagret                                                                                       & Medel & 4h         &     \\ \hline
Egen studier & Läs kurslitteratur för inspiration                                                                       & Medel
& 4h         & \\  \hline
\end{tabularx}
\end{table}

Kommentar:

\pagebreak

\hphantom{$\bullet$} \textbf{Vecka 39}\\

Hård deadline

\begin{itemize}
\color{red}
    \item 29/9 Eventuella korregeringar till projektplan, installationsmanual eller Git

    \item 30/9 Datalagret godkänt av assistent
\end{itemize}

\begin{table}[h]
{\centering\begin{tabularx}{\textwidth}{|l|X|l|l|l|}
  \hline
Aktivitet            & Beskrivning                                                                                                              & Prioritet  & Avsatt tid & Tid \\  \hline
Datalagret           & Slutföra och godkännas av assistent                                                                                      & Hög   & 4h         &         \\  \hline
Korrigeringar        & Eventuellt korrigera datalagret, installationsmanualen projektplan eller Git  & Hög   & 4,5h       &     \\  \hline
Egen studier         & Läs kurslitteratur för inspiration                                                                                       & Hög   & 3h         &     \\  \hline
Systemdokumentation & Skapa och påbörja dokumentation                                                                                          & Medel & 2h         &    \\
 \hline
\end{tabularx}\par}
\end{table}

Kommentar:

\pagebreak

\hphantom{$\bullet$} \textbf{Vecka 40}\\

Intern deadline
\begin{itemize}
\color{blue}
    \item Skicka in första version av systemdokumentation
\end{itemize}

\begin{table}[h]
\begin{tabularx}{\textwidth}{|l|X|l|l|l|}
  \hline
Aktivitet            & Beskrivning                             & Prioritet  & Avsatt tid & Tid \\  \hline
Korrigeringar        & Eventuellt korrigeringar till projektet & Hög   & 5h         &     \\  \hline
Systemdokumentation & Skapa och påbörja dokumentation         & Hög   & 2h         &     \\  \hline
Testdokumentation    & Skapa och påbörja                       & Medel & 2h         &     \\  \hline
Reflektionsdokument  & Skapa och påbörja                       & Låg   & 1h         &   \\   \hline
\end{tabularx}
\end{table}

Kommentar:

\pagebreak

\hphantom{$\bullet$} \textbf{Vecka 41}\\

Hård deadline

\begin{itemize}
\color{red}
    \item 13/10 Systemdemonstration för andra grupper, grund till

testdokumentation

    \item 13/10 Portfolio publicerad

    \item 13/10 Första version av systemdokumentation inlämnad

\end{itemize}

\begin{table}[!h]
\begin{tabularx}{\textwidth}{|l|X|l|l|l|}
  \hline
Aktivitet            & Beskrivning                             & Prioritet  & Avsatt tid & Tid \\  \hline
Korrigeringar        & Eventuellt korrigeringar till projektet & Hög   & 4h         &     \\  \hline
Systemsdokumentation & Första version inlämnad                 & Hög   & 3h         &     \\  \hline
Testdokumentation    & Skapa och påbörja                       & Hög   & 2h         &     \\  \hline
Reflektionsdokument  & Fortsatt dokumentation                  & Medel & 2h         &    \\   \hline
\end{tabularx}
\end{table}
Kommentarer:

\pagebreak

\hphantom{$\bullet$} \textbf{Vecka 42}\\

\begin{table}[h]
\begin{tabularx}{\textwidth}{|l|X|l|l|l|}
  \hline
Aktivitet           & Beskrivning                                      & Prioritet & Avsatt tid & Tid \\  \hline
Testdokumentation   & Slutföra och lämna in                            & Hög  & 2h         &     \\  \hline
Reflektionsdokument & Slutföra och lämna in                            & Hög  & 2h         &     \\  \hline
Korrigeringar       & Ändra eventuella brister i
systemdokumentationen & Hög  & 3h         &
\\   \hline
\end{tabularx}
\end{table}

Kommentarer:

\end{document}
